We show how to setup the assembly programming
environment for the arduino.
%
\begin{enumerate}[label=\arabic*.,ref=\theenumi]
\item Copy the .inc file to your home directory
\begin{lstlisting}
cp assembly/setup/m328Pdef/m328Pdef.inc ~/
\end{lstlisting}
\item Execute
\begin{lstlisting}
avra assembly/setup/codes/hello.asm
\end{lstlisting}
		Make sure that the path to \textbf{m328Pdef.inc} is correctly given in \textbf{hello.asm}.
\item Then  flash the .hex file
\begin{lstlisting}
hello.hex
\end{lstlisting}
\item You should
see the led beside pin 13 light up.
\item Now edit \textbf{hello.asm} by modifying the line to
\begin{lstlisting}
ldi r17,0b00000000
\end{lstlisting}
Save and execute.  The led should turn off.
\item What do the following instructions do?
\begin{lstlisting}
ldi r16,0b00100000
out DDRB,r16
\end{lstlisting}
\solution The Atmega328p microcontroller for the arduino board has 32 internal 8-bit registers, R0-R31. R16-R31 can be used directly for i/o.  The first instruction loads an 8-bit binary number into  R16. The second instruction loads the value in R16 to the DDRB register.  Each bit of the DDRB register corresponds to a pin on the arduino. The second instruction declares pin 13 to be an output port. Both the instructions are equivalent to  pinMode(13, OUTPUT).  
\item What do the following instructions do?
\begin{lstlisting}
  ldi r17,0b00100000
  out PortB,r17
\end{lstlisting}
\solution The instructions are equivalent to digitalWrite(13).


\end{enumerate}



