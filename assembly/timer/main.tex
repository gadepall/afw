We show how to use the Atmega328p timer to blink the builtin led with a delay.
\begin{enumerate}[label=\arabic*.,ref=\theenumi]
\item Connect the Arduino to the computer and execute the following code
\begin{lstlisting}
assembly/timer/codes/timer.asm
\end{lstlisting}

\item Explain the following instruction
\begin{lstlisting}
sbi DDRB, 5
\end{lstlisting}
\item What do the following instructions do?
\begin{lstlisting}
ldi r16, 0b00000101 
out TCCR0B, r16 
\end{lstlisting}
\solution The system clock (SYSCLK) frequency of the Atmega328p is 16 MHz.  TCCR0B is the Timer Counter Control Register.  When 
\begin{align}
TCCR0B&=0b101 \\
\implies CLK &= \frac{SYSCLK}{1024} \\
&= \frac{16M}{1K} = 16kHz.
\end{align}
\item Explain the PAUSE routine.
\begin{lstlisting}
ldi r19, 0b01000000 ;times to run the loop = 64 for 1 second delay
PAUSE:	;this is delay (function)
lp2:	;loop runs 64 times
		IN r16, TIFR0 ;tifr is timer interupt flag (8 bit timer runs 256 times)
		ldi r17, 0b00000010
		AND r16, r17 ;need second bit
		BREQ PAUSE 
		OUT TIFR0, r17	;set tifr flag high
	dec r19
	brne lp2
	ret
\end{lstlisting}
\solution TIFR0 is the timer interrupt flag and TIFR0=0bxxxxxx10 after every 256 cycles. PAUSE routine waits till TIFR0=0bxxxxxx10, this checking is done by the AND and BREQ instructions above.  
\item Explain the lp2 routine.
\\
\solution R19 = 64 and is used as a count for lp2. The lp2 routine returns after 64 PAUSE rutines.
\item What is the blinking delay?
\\
\solution The blinking delay is given by
\begin{align}
delay &= \frac{CLK}{lp2\times PAUSE} seconds
\\
& = \frac{16 \times 1024}{64 \times 256} seconds = 1 second
\end{align}
\end{enumerate}
%
\subsection{Blink through Cycle Delays}
\begin{enumerate}[label=\arabic*.,ref=\theenumi]
\item Connect pin 8 of the Arduino to an led and execute the following code
\begin{lstlisting}
assembly/timer/codes/cycle_delay.asm
\end{lstlisting}
\item Explain how the delay is obained
\begin{lstlisting}
    ldi r16,0x50
    ldi r17,0x00
    ldi r18,0x00

w0:
    dec r18
    brne w0
    dec r17
    brne w0
    dec r16
    brne w0
    pop r18
    pop r17
    pop r16
    ret
\end{lstlisting}
\solution The w0 loop is executed using the counts in R16=$2^6+2^4 = 80$, R17=R18=$2^8=256$.  Thus 
\begin{align}
delay &\approx 80\times256\times256 cycles
\\
&= \frac{80\times256\times256 }{2^4 \times 2^20} seconds
\\
&=0.3125 seconds
\end{align}
The actual time is slightly more since each instruction takes a few cycles to execute. 
\item Should you use timer delay or cycle delay?
\\
\solution Timer delay is an accurate method for giving delays.  Cycle delay is a crude method and should be avoided.  
\end{enumerate}


